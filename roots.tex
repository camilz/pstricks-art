\documentclass[a4paper]{article}

\usepackage{multido}
\usepackage{pst-all} % pst-tree, pst-node}

\begin{document}

% Radius \R of the circle
\def\R{10}

% \r is used for dimensioning the pstricks environment
\pstFPdiv{\r}{\R}{2}
\pstFPstripZeros{\r}{\r}

% Number of roots \N of the unity
\def\N{720}

% Number used in multiplication
\def\NC{13}

% Calculate the angle increments
% dividing 360 degrees by \n
\pstFPdiv{\AI}{360}{\N}

% The calculations are permormed
% in greater precision that pstricks
% is able to deal, hence we strip
% the zeros at right
\pstFPstripZeros{\AI}{\AI}

\begin{figure}[b]
\centering
\begin{pspicture}(-\r,-\r)(\r,\r)

% Draws a circle of radius \R
% and centered in (0,0)
\pscircle(0,0){\R}

% Calculate the angle of this point and save in \AP
\pstFPmul{\AP}{\i}{\AI}
\pstFPstripZeros{\AP}{\AP}

\multido{\i=1+1}{\N}{
% For each point
\pstFPmul{\AP}{\i}{\AI}
% Strip zeros from \AP
\pstFPstripZeros{\AP}{\AP}
% Create a point A. The trick is to create this point
% at coordinates (\R,0) inside the \psrotate command
% for displaying it where we want it to be.
\psrotate(0,0){\AP}{\pnode[\R,0]{A}}
% Here we implement our rule for joining the points
% by a straight line.
\pstFPmul{\BA}{\NC}{\AP}
% Strip zeros from \BA
\pstFPstripZeros{\BA}{\BA}
% Also rotate the \pnode by \BA degrees.
\psrotate(0,0){\BA}{\pnode[\R,0]{B}}
% Finally, joing the nodes A and B
\ncline{A}{B}
} % End Multido

\end{pspicture}
\end{figure}

\end{document}
